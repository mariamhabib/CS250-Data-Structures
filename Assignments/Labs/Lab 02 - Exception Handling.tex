\documentclass[a4paper,12pt]{book}
\usepackage[utf8]{inputenc}
\title{}
\author{Rachel Morris}
\date{\today}

\usepackage{rachwidgets}
\usepackage{fancyhdr}
\usepackage{lastpage}
\usepackage{boxedminipage}

\pagestyle{fancy}
\fancyhf{}
\lhead{CS 250}
\chead{Fall 2017}
\rhead{Lab 1: STL Containers}
\rfoot{\thepage\ of \pageref{LastPage}}
\lfoot{By Rachel Morris \date}

\renewcommand{\headrulewidth}{2pt}
\renewcommand{\footrulewidth}{1pt}

\begin{document}

    \chapter*{Lab 2: Exception Handling} \stepcounter{chapter}

        \section*{Information}
            \paragraph{ Topics: } Exception handling, file streams
            \paragraph{ Turn in: } All source files (.cpp and .hpp), question document as .txt file. 


% ----------------------------------------------------------------------
% ----------------------------------------------------------------------
% ----------------------------------------------------------------------
        \section*{Getting started}

            You will start with the following code, which contains
            an object that handles file streams, as well as the main
            source file that contains tests.
            At the moment, it should compile and run, but it will crash
            during the tests because the RecordManager object doesn't
            have error checking.

            In this lab, you will implement the exception handling
            within the RecordManager, as well as checking for exceptions
            from the main file.

            ~\\
            The following code is the starter code that you will work with;
            you can download the starter code, or type it out manually.
            There are descriptions for how the functions work.

            Once entered, the next part of the lab is to add the
            error checking and exception handling.

            \newpage{}
            \subsection*{Starter code}
            \subsubsection*{lab2\_RecordManager.hpp}

            The RecordManager contains an array of output-file-streams
            (\texttt{ofstream}). When the \texttt{OpenOutputFile} function
            is called to open a new file, a free spot in the array is found
            and its index is used for opening the file, writing to the
            file, and closing the file.

\begin{lstlisting}[style=code]
#ifndef _RECORD_MANAGER_HPP
#define _RECORD_MANAGER_HPP

#include <string>
#include <iostream>
#include <fstream>
#include <stdexcept>
using namespace std;

class RecordManager
{
    public:
    ~RecordManager();
    
    void OpenOutputFile( string filename );
    void CloseOutputFile( string filename );
    void WriteToFile( string filename, string text );
    void DisplayAllOpenFiles();
    
    private:
    int FindAvailableFile();
    int FindFilenameIndex( string filename );
    
    ofstream m_outputs[5];
    string m_filenames[5];
    const int MAX_FILES = 5;
};

#endif
\end{lstlisting}

            \newpage
            \subsubsection*{lab2\_RecordManager.cpp}

            \paragraph{Destructor}
            The destructor \texttt{\textasciitilde RecordManager} is
            a function that gets called automatically once the object
            is destroyed. This is a good place to close any currently-open
            file streams, if any are open.

            \begin{hint}{Include the hpp file!}
                Don't forget to include the header file!
\begin{verbatim}
#include "lab2_RecordManager.hpp"
\end{verbatim}
            \end{hint}
            
\begin{lstlisting}[style=code]
RecordManager::~RecordManager()
{
    for ( int i = 0; i < MAX_FILES; i++ )
    {
        if ( m_outputs[i].is_open() )
        {
            m_outputs[i].close();
        }
    }
}
\end{lstlisting}

            \paragraph{OpenOutputFile}
            When the user wants to open a file, they pass in the filename
            to open. Via the member function, \texttt{FindAvailableFile},
            an available index is found. Then, the file strema at that
            position is opened for usage.

            If the \texttt{FindAvailableFile} function cannot find
            an available index (that is, all files in the array are
            currently open), then it will return -1. For this function,
            trying to access \texttt{m\_outputs[ index ]} will cause
            the program to crash.
            
\begin{lstlisting}[style=code]
void RecordManager::OpenOutputFile( string filename )
{
    int index = FindAvailableFile();
    m_outputs[ index ].open( filename );
    m_filenames[ index ] = filename;
}
\end{lstlisting}

            \paragraph{CloseOutputFile}
            This function receives the filename that it should close.
            Then, this function calls \texttt{FindFilenameIndex} to
            search through the array of filenames, and if it is found,
            the index is returned. If the filename is not found in the array,
            then -1 is returned. Again, if \texttt{m\_outputs[ index ]}
            is accessed when index is -1, the program will crash.

\newpage
\begin{lstlisting}[style=code]
void RecordManager::CloseOutputFile( string filename )
{
    int index = FindFilenameIndex( filename );
    m_outputs[ index ].close();
    m_filenames[ index ] = "";
}
\end{lstlisting}

            \paragraph{WriteToFile}
            This function takes in a filename to be written to,
            as well as some text to write out. Like with \texttt{CloseOutputFile},
            it uses \texttt{FindFilenameIndex} to find an index to work with.
            Then, it writes the \texttt{text} to that output file.
            
\begin{lstlisting}[style=code]
void RecordManager::WriteToFile( string filename, string text )
{
    int index = FindFilenameIndex( filename );
    m_outputs[ index ] << text << endl;
}
\end{lstlisting}

            \paragraph{DisplayAllOpenFiles}
            This function iterates through the list of filenames
            and outputs them, if the corresponding file stream is open.

            \textbf{This function doesn't cause a crash.}
            
\begin{lstlisting}[style=code]
void RecordManager::DisplayAllOpenFiles()
{
    cout << "Open files: " << endl;
    for ( int i = 0; i < MAX_FILES; i++ )
    {
        if ( m_outputs[i].is_open() )
        {
            cout << i << ". " << m_filenames[i] << endl;
        }
    }
}
\end{lstlisting}

            \newpage
            \paragraph{FindAvailableFile}
            This function iterates through all the files, and for the first
            element it finds that isn't open, it will return its index
            for usage by other functions in the class. If all files
            are open (none are available), -1 is returned.

            \textbf{This function doesn't cause a crash.}
            
\begin{lstlisting}[style=code]
int RecordManager::FindAvailableFile()
{
    for ( int i = 0; i < MAX_FILES; i++ )
    {
        if ( m_outputs[i].is_open() == false )
        {
            return i;
        }
    }
    return -1;
}
\end{lstlisting}

            \paragraph{FindFilenameIndex}
            This function will search for a filename in the array
            that matches the one passed in as input. If that filename
            is found, it returns the corresponding index. Otherwise,
            it returns -1.

            \textbf{This function doesn't cause a crash.}
            
\begin{lstlisting}[style=code]
int RecordManager::FindFilenameIndex( string filename )
{
    for ( int i = 0; i < MAX_FILES; i++ )
    {
        if ( m_filenames[i] == filename )
        {
            return i;
        }
    }
    return -1;
}
\end{lstlisting}

            \newpage
            \subsubsection*{lab2\_main.cpp}

            The main file contains 5 tests, each called from main().
            Initially, the program will crash while running the tests,
            but as you implement the error checking and exception
            handling, it should run through all 5 tests.

\begin{lstlisting}[style=code]
#include "lab2_RecordManager.hpp"

#include <iostream>
using namespace std;

void Test1()
{
    cout << endl << "-------------------------------------" << endl;
    cout << "TEST 1: Open one file and write to it" << endl << endl;
    RecordManager record;
    record.OpenOutputFile( "Test1.txt" );
    
    record.DisplayAllOpenFiles();
    
    record.WriteToFile( "Test1.txt", "Hello world!" );
    
    record.CloseOutputFile( "Test1.txt" );
    
    cout << endl << "END OF TEST 1" << endl;
}

void Test2()
{
    cout << endl << "-------------------------------------" << endl;
    cout << "TEST 2: Open 5 files and write to them" << endl << endl;
    RecordManager record;
    record.OpenOutputFile( "Test2_A.txt" );
    record.OpenOutputFile( "Test2_B.txt" );
    record.OpenOutputFile( "Test2_C.txt" );
    record.OpenOutputFile( "Test2_D.txt" );
    record.OpenOutputFile( "Test2_E.txt" );
    
    record.DisplayAllOpenFiles();
    
    record.WriteToFile( "Test2_A.txt", "ABCDE" );
    record.WriteToFile( "Test2_B.txt", "FGHIJ" );
    record.WriteToFile( "Test2_C.txt", "KLMNO" );
    record.WriteToFile( "Test2_D.txt", "PQRST" );
    record.WriteToFile( "Test2_E.txt", "UVWXYZ" );
    
    record.CloseOutputFile( "Test2_A.txt" );
    record.CloseOutputFile( "Test2_B.txt" );
    record.CloseOutputFile( "Test2_C.txt" );
    record.CloseOutputFile( "Test2_D.txt" );
    record.CloseOutputFile( "Test2_E.txt" );
    
    cout << endl << "END OF TEST 2" << endl;
}

void Test3()
{
    cout << endl << "-------------------------------------" << endl;
    cout << "TEST 3: Write to a file that isn't opened" << endl << endl;
    RecordManager record;
    
    record.DisplayAllOpenFiles();
    
    record.WriteToFile( "Test2.txt", "How are you?" );
    
    cout << endl << "END OF TEST 3" << endl;
}

void Test4()
{
    cout << endl << "-------------------------------------" << endl;
    cout << "TEST 4: Close a file that isn't opened" << endl << endl;
    RecordManager record;
    
    record.DisplayAllOpenFiles();
    
    record.CloseOutputFile( "Test3.txt" );

    cout << endl << "END OF TEST 4" << endl;
}

void Test5()
{
    cout << endl << "-------------------------------------" << endl;
    cout << "TEST 5: Try to open more than max # of files" << endl << endl;
    RecordManager record;

    record.OpenOutputFile( "Test5_A.txt" );
    record.OpenOutputFile( "Test5_B.txt" );
    record.OpenOutputFile( "Test5_C.txt" );
    record.OpenOutputFile( "Test5_D.txt" );
    record.OpenOutputFile( "Test5_E.txt" );
    record.OpenOutputFile( "Test4_F.txt" ); // too many

    record.DisplayAllOpenFiles();

    cout << endl << "END OF TEST 5" << endl;
}

int main()
{
    Test1();
    Test2();
    Test3();
    Test4();
    Test5();
    
    
    return 0;
}
\end{lstlisting}

% ----------------------------------------------------------------------
% ----------------------------------------------------------------------
% ----------------------------------------------------------------------
        \newpage
        \section*{Adding error checking}

            Within the RecordManager class functions, you will add error
            checking and \texttt{throw} commands. Outside the class, in
            main and the tests, you will add the \texttt{try/catch} blocks.
            
            \begin{intro}{Throw, try, and catch}
                \textbf{INSIDE} a function is where the
                \texttt{throw} command is used. After checking
                for some error state, the response is to
                \texttt{throw} the type of exception that has
                occurred, as well as an error message. 
                ~\\~\\
                \textbf{OUTSIDE} a function (at the function-call level)
                is where \texttt{try} and \texttt{catch} commands
                are used. When a function that may throw an exception is
                being called, it should be wrapped within a \texttt{try}
                block. Then, the \texttt{catch} block(s) follow, to catch
                different types of exceptions, resolve them, clean up,
                and allow the program to continue.
            \end{intro}
            
            \subsubsection*{Updating RecordManager::OpenOutputFile}
            In this function, add an error check before
            \texttt{m\_outputs[ index ]} is accessed: Check to see
            if the \textit{index} value is equal to -1. If so, throw
            a \texttt{runtime\_error} like this:

\begin{lstlisting}[style=code]
if ( index == -1 )
{
    throw runtime_error( "No available files" );
}    
\end{lstlisting}

            \subsubsection*{Updating RecordManager::CloseOutputFile}
            This function should also throw a \texttt{runtime\_error} if
            the returned index is -1.

            \subsubsection*{Updating RecordManager::WriteToFile}
            This function should also throw a \texttt{runtime\_error} if
            the returned index is -1.

            \subsubsection*{Updating RecordManager::DisplayAllOpenFiles}
            This function won't cause any exceptions. Therefore,
            you should add the function specifier, \texttt{noexcept},
            to the end of the signature - both in the declaration and
            the definition.

            \subsubsection*{Updating RecordManager::FindAvailableFile}
            Add the \texttt{noexcept} function specifier.

            \subsubsection*{Updating RecordManager::FindFilenameIndex}
            Add the \texttt{noexcept} function specifier.

            \hrulefill
            \subsubsection*{Testing it out}

            Now when you run the program, it won't flat out crash like
            before, but it will abort the program once the first exception
            is hit.

\begin{framed}
\begin{lstlisting}[style=output]
-------------------------------------
TEST 1: Open one file and write to it

Open files: 
0. Test1.txt

END OF TEST 1

-------------------------------------
TEST 2: Open 5 files and write to them

Open files: 
0. Test2_A.txt
1. Test2_B.txt
2. Test2_C.txt
3. Test2_D.txt
4. Test2_E.txt

END OF TEST 2

-------------------------------------
TEST 3: Write to a file that isn't opened

Open files: 
terminate called after throwing an instance of 'std::runtime_error'
  what():  File Test2.txt is not open
Aborted
\end{lstlisting}
\footnotesize (Note: This is the example output for the program running in Linux,
with the g++ compiler. The result text might look different in Visual Studio.)
\end{framed}

            Next, \texttt{try/catch} blocks will need to be added
            in our tests so that the program will complete its execution,
            even if exceptions are discovered.

% ----------------------------------------------------------------------
% ----------------------------------------------------------------------
% ----------------------------------------------------------------------
        \newpage
        \section*{Adding try/catch}

            When we are calling a function that may cause an exception,
            that's when we should have a \texttt{try/catch} statement.
            The functions from \texttt{RecordManager} that may cause
            exceptions are:

            \begin{itemize}
                \item \texttt{OpenOutputFile}
                \item \texttt{CloseOutputFile}
                \item \texttt{WriteToFile}
            \end{itemize}

            Test1 and Test2 themselves won't cause any exceptions to be
            thrown, but if we were to add the try/catch to Test1, it would
            look like this:
            
\begin{lstlisting}[style=code]
// (...)
try
{
    record.DisplayAllOpenFiles();
    
    record.WriteToFile( "Test1.txt", "Hello world!" );
    
    record.CloseOutputFile( "Test1.txt" );
}
catch( runtime_error& ex )
{
    cout << "Error: " << ex.what() << endl;
}
// (...)
\end{lstlisting}

            ~\\
            We could wrap each individual function in a try/catch,
            or wrap all three of them together. This is a design decision
            to make. In general, it is considered good design to wrap your
            try/catch around the smallest possible amount of code,
            but it also doesn't need to wrap just one line at a time.
            Since these three functions are related, we can wrap the three
            in a single try/catch block and it would be fairly clean.

            Mainly, don't wrap an entire function in a try/catch -
            only a section working with ``risky" areas.

            \subsection*{Updating Test3, Test4, and Test5}
            For each of these tests, surround only the function calls
            that may return with an exception. Remember that any
            functions marked as \texttt{noexcept} will never throw
            an exception.

            Once you've updated the entire program, the output should
            look like this:

\begin{lstlisting}[style=output]
-------------------------------------
TEST 1: Open one file and write to it

Open files: 
0. Test1.txt

END OF TEST 1

-------------------------------------
TEST 2: Open 5 files and write to them

Open files: 
0. Test2_A.txt
1. Test2_B.txt
2. Test2_C.txt
3. Test2_D.txt
4. Test2_E.txt

END OF TEST 2

-------------------------------------
TEST 3: Write to a file that isn't opened

Open files: 
Error: File Test2.txt is not open

END OF TEST 3

-------------------------------------
TEST 4: Close a file that isn't opened

Open files: 
Error: File Test3.txt is not open

END OF TEST 4

-------------------------------------
TEST 5: Try to open more than max # of files

Error: No available files
Open files: 
0. Test5_A.txt
1. Test5_B.txt
2. Test5_C.txt
3. Test5_D.txt
4. Test5_E.txt

END OF TEST 5
\end{lstlisting}

           
% ----------------------------------------------------------------------
% ----------------------------------------------------------------------
% ---------------------------------------------------------------------- 
        \newpage
        \subsection*{Questions and Answers}
            Answer the following questions in a .txt file and
            turn them in with the rest of the lab.
            You will also want to reference the exception handling
            lecture for these questions.

            \begin{enumerate}
                \item What are the four levels of exception safety?
                \item How many \texttt{catch} blocks can there be
                    with a single \texttt{try} block?
                \item Which exceptions are children of the \texttt{logic\_error} exception?
                \item Which exceptions are children of the \texttt{runtime\_error} exception?
            \end{enumerate}

\end{document}










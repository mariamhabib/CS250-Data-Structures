\documentclass[a4paper,12pt,oneside]{book}
\usepackage[utf8]{inputenc}
\title{}
\author{Rachel Morris}
\date{\today}

\usepackage{rachwidgets}
\usepackage{fancyhdr}
\usepackage{lastpage}
\usepackage{boxedminipage}
\usepackage{fancyvrb}

\newcommand{\laClass}{CS 250\ }
\newcommand{\laSemester}{Fall 2017\ }
\newcommand{\laLab}{Lab 10: Stacks and Queues\ }
\newcounter{question}

\pagestyle{fancy}
\fancyhf{}
\lhead{\laClass}
\chead{\laSemester}
\rhead{\laLab}
\rfoot{\thepage\ of \pageref{LastPage}}
\lfoot{By Rachel Morris, \footnotesize last updated \today}

\renewcommand{\headrulewidth}{2pt}
\renewcommand{\footrulewidth}{1pt}

%\toggletrue{answerkey}
\togglefalse{answerkey}

\begin{document}

    \chapter*{\laLab} \stepcounter{chapter}

        \section{Information}
            \paragraph{ Topics: } Stacks, Queues
            \paragraph{ Turn in: } Your work in a text file, PDF file, scanned or photographed images.
            
% ----------------------------------------------------------------------
% ----------------------------------------------------------------------
% ----------------------------------------------------------------------

~\\
This is another on-paper ``lab", which is meant to help you become
more familiar with how stacks and queues work.

    \hrulefill

    \section{About: Queues}

        Queues are first-in-first-out (FIFO) structures. The main
        functionality of a Queue is \textbf{Push} (to the back),
        \textbf{Pop} (from the front), and \textbf{Front}, which accesses the first
        item in the queue.

        Since items are pushed into the back of the queue and pulled out
        from the front, we essentially will work with the item that has
        been waiting the longest period of time before any of the items
        that were added \textit{after it} get removed.

        \begin{center}
            \begin{tabular}{ | p{2cm} | p{2cm} | p{2cm} | p{2cm} | p{2cm} |}
                \hline
                0 & 1 & 2 & 3 & 4
                \\ \hline
                FRONT & & & & BACK
                \\ \hline
            \end{tabular}
        \end{center}

        
    \newpage
    \section{About: Stacks}

        Stacks are last-in-first-out (LIFO) structures. The main
        functionality of a Queue is \textbf{Push} (to the back),
        \textbf{Pop} (from the back), and \textbf{Top}, which accesses the
        last item in the stack (aka the ``top" of the stack).

        Stacks are usually thought of as items being stacked vertically,
        but if you're thinking of an array in a horizontal manner, the ``bottom"
        is at index 0, and the ``top" is at the index $n-1$, for a stack of size $n$.

        Items are both pushed into the back and popped from the back, so
        before the ``bottom-most" (first-most) item is removed, all items
        added \textit{after it} must be removed first.

        \begin{center}
            \begin{tabular}{ | p{2cm} | p{2cm} | p{2cm} | p{2cm} | p{2cm} |}
                \hline
                0 & 1 & 2 & 3 & 4
                \\ \hline
                BOTTOM & & & & TOP
                \\ \hline
            \end{tabular}
        \end{center}

        Or...

        \begin{center}
            \begin{tabular}{ | c | c |}
                \hline
                4 & TOP
                \\ \hline
                3 &
                \\ \hline
                2 &
                \\ \hline
                1 &
                \\ \hline
                0 & BOTTOM
                \\ \hline
            \end{tabular}
        \end{center}

    \newpage

    \section{Questions: Queues}

        % -------------------------------------------------------------%
        % - QUESTION --------------------------------------------------%
        % -------------------------------------------------------------%
        \stepcounter{question}
        \begin{question}{\thequestion}{2}
            Draw a \textbf{queue} structure at each step:

            \begin{enumerate}
                \item   \begin{verbatim} Push( "A" ); \end{verbatim}
                    \solution{
                        \begin{tabular}{ | c | }
                            A
                        \end{tabular}
                    }{ { ~\\ \raisebox{0pt}[0.5cm][0pt]{  } } }
                
                \item   \begin{verbatim} Push( "B" ); \end{verbatim}
                    \solution{
                        \begin{tabular}{ | c | c | }
                            A & B
                        \end{tabular}
                    }{ { ~\\ \raisebox{0pt}[0.5cm][0pt]{  } } }
                
                \item   \begin{verbatim} Push( "C" ); \end{verbatim}
                    \solution{
                        \begin{tabular}{ | c | c | c | }
                            A & B & C
                        \end{tabular}
                    }{ { ~\\ \raisebox{0pt}[0.5cm][0pt]{  } } }

                \item   \begin{verbatim} Pop(); \end{verbatim}
                    \solution{
                        \begin{tabular}{ | c | c | }
                            B & C
                        \end{tabular}
                    }{ { ~\\ \raisebox{0pt}[0.5cm][0pt]{  } } }
                
                \item   \begin{verbatim} Push( "D" ); \end{verbatim}
                    \solution{
                        \begin{tabular}{ | c | c | c | }
                            B & C & D 
                        \end{tabular}
                    }{ { ~\\ \raisebox{0pt}[0.5cm][0pt]{  } } }
                
            \end{enumerate}
        \end{question}

        % -------------------------------------------------------------%
        % - QUESTION --------------------------------------------------%
        % -------------------------------------------------------------%
        \stepcounter{question}
        \begin{question}{\thequestion}{2}
            For the following queue, write out code commands for the queue
            (e.g., Push(``..."), Pop(), or Front()) to access element ``D". Write in a linear
            order. At the end of your steps, the Front() function should return
            the requested item.

            \begin{center}
                \begin{tabular}{ | c | c | c | c | c | c | }
                    \hline
                    0 & 1 & 2 & 3 & 4 & 5
                    \\ \hline
                    A & S & D & F & G & H
                    \\ \hline
                \end{tabular}
            \end{center}

            \solution{Pop() Pop() Front()}{}
        \end{question}

        \newpage
        % -------------------------------------------------------------%
        % - QUESTION --------------------------------------------------%
        % -------------------------------------------------------------%
        \stepcounter{question}
        \begin{question}{\thequestion}{2}
            For each set of commands, what will be returned by the Front()
            function after all the commands have been executed?

            \begin{itemize}
                \item[a.]   \texttt{Push('A'); Push('B'); Push('C');}
                    ~\\~\\
                    \texttt{Front()} returns \solution{'A'}{ \fitb[2cm] }
                ~\\
                \item[b.]   \texttt{Push(2); Push(3); Push(4); Pop(); Push(5); Push(6);}
                    ~\\~\\
                    \texttt{Front()} returns \solution{3}{ \fitb[2cm] }
                ~\\
                \item[c.]   \texttt{Push(0); Pop(); Push(1); Push(2); Pop(); Push(3); Push(4);}
                    ~\\~\\
                    \texttt{Front()} returns \solution{2}{ \fitb[2cm] }
            \end{itemize}
        \end{question}

        \hrulefill
        % -------------------------------------------------------------%
        % - QUESTION --------------------------------------------------%
        % -------------------------------------------------------------%
        \stepcounter{question}
        \begin{question}{\thequestion}{1}
            Give at least one example of how a queue may be used in computer science.
            Try to find an example online if you can't think of anything.
            If you can't find any good examples of software uses for programming,
            think of other technology/engineering fields and given an example.
        \end{question}

        
    \newpage
    \section{Questions: Stacks}

        % -------------------------------------------------------------%
        % - QUESTION --------------------------------------------------%
        % -------------------------------------------------------------%
        \stepcounter{question}
        \begin{question}{\thequestion}{2}
            Draw a \textbf{stack} structure at each step:

            \begin{enumerate}
                \item   \begin{verbatim} Push( "A" ); \end{verbatim}
                    \solution{
                        \begin{tabular}{ | c | }
                            A
                        \end{tabular}
                    }{ { ~\\ \raisebox{0pt}[0.5cm][0pt]{  } } }
                
                \item   \begin{verbatim} Push( "B" ); \end{verbatim}
                    \solution{
                        \begin{tabular}{ | c | c | }
                            A & B
                        \end{tabular}
                    }{ { ~\\ \raisebox{0pt}[0.5cm][0pt]{  } } }
                
                \item   \begin{verbatim} Push( "C" ); \end{verbatim}
                    \solution{
                        \begin{tabular}{ | c | c | c | }
                            A & B & C
                        \end{tabular}
                    }{ { ~\\ \raisebox{0pt}[0.5cm][0pt]{  } } }

                \item   \begin{verbatim} Pop(); \end{verbatim}
                    \solution{
                        \begin{tabular}{ | c | c | }
                            B & C
                        \end{tabular}
                    }{ { ~\\ \raisebox{0pt}[0.5cm][0pt]{  } } }
                
                \item   \begin{verbatim} Push( "D" ); \end{verbatim}
                    \solution{
                        \begin{tabular}{ | c | c | c | }
                            B & C & D 
                        \end{tabular}
                    }{ { ~\\ \raisebox{0pt}[0.5cm][0pt]{  } } }
                
            \end{enumerate}
        \end{question}
        
        % -------------------------------------------------------------%
        % - QUESTION --------------------------------------------------%
        % -------------------------------------------------------------%
        \stepcounter{question}
        \begin{question}{\thequestion}{2}
            For the following stack, write out code commands for the queue
            (e.g., Push(``..."), Pop(), or Top()) to access element ``D". Write in a linear
            order. At the end of your steps, the Front() function should return
            the requested item.

            \begin{center}
                \begin{tabular}{ | c | c | c | c | c | c | }
                    \hline
                    0 & 1 & 2 & 3 & 4 & 5
                    \\ \hline
                    A & S & D & F & G & H
                    \\ \hline
                \end{tabular}
            \end{center}

            \solution{Pop() Pop() Pop() Top()}{}
        \end{question}


        \newpage
        % -------------------------------------------------------------%
        % - QUESTION --------------------------------------------------%
        % -------------------------------------------------------------%
        \stepcounter{question}
        \begin{question}{\thequestion}{2}
            For each set of commands, what will be returned by the Top()
            function after all the commands have been executed?

            \begin{itemize}
                \item[a.]   \texttt{Push('A'); Push('B'); Push('C');}
                    ~\\~\\
                    \texttt{Top()} returns \solution{'A'}{ \fitb[2cm] }
                ~\\
                \item[b.]   \texttt{Push(2); Push(3); Push(4); Pop(); Push(5); Push(6);}
                    ~\\~\\
                    \texttt{Top()} returns \solution{3}{ \fitb[2cm] }
                ~\\
                \item[c.]   \texttt{Push(0); Pop(); Push(1); Push(2); Pop(); Push(3); Push(4);}
                    ~\\~\\
                    \texttt{Top()} returns \solution{2}{ \fitb[2cm] }
            \end{itemize}
        \end{question}

        \hrulefill
        % -------------------------------------------------------------%
        % - QUESTION --------------------------------------------------%
        % -------------------------------------------------------------%
        \stepcounter{question}
        \begin{question}{\thequestion}{1}
            Give at least one example of how a stack may be used in computer science.
            Try to find an example online if you can't think of anything.
            If you can't find any good examples of software uses for programming,
            think of other technology/engineering fields and given an example.
        \end{question}

        
\end{document}










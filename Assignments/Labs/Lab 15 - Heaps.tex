\documentclass[a4paper,12pt,oneside]{book}
\usepackage[utf8]{inputenc}
\title{}
\author{Rachel Morris}
\date{\today}

\usepackage{rachwidgets}
\usepackage{fancyhdr}
\usepackage{lastpage}
\usepackage{boxedminipage}
\usepackage{fancyvrb}

\newcommand{\laClass}{CS 250\ }
\newcommand{\laSemester}{Fall 2017\ }
\newcommand{\laLab}{Lab 15: Heaps\ }
\newcounter{question}

\pagestyle{fancy}
\fancyhf{}
\lhead{\laClass}
\chead{\laSemester}
\rhead{\laLab}
\rfoot{\thepage\ of \pageref{LastPage}}
\lfoot{By Rachel Morris, \footnotesize last updated \today}

\renewcommand{\headrulewidth}{2pt}
\renewcommand{\footrulewidth}{1pt}

%\toggletrue{answerkey}
\togglefalse{answerkey}

\begin{document}

    \chapter*{\laLab} \stepcounter{chapter}

        \section{Information}
            \paragraph{ Topics: } Heaps
            \paragraph{ Turn in: } This is another on-paper ``lab". Turn in a text file, PDF file, scanned or photographed images.
            
% ----------------------------------------------------------------------
% ----------------------------------------------------------------------
% ----------------------------------------------------------------------

    \hrulefill

    \section{About: Heaps}

    A Heap is a specialized type of binary tree structure. 

    \hrulefill

    % -------------------------------------------------------------%
    % - QUESTION --------------------------------------------------%
    % -------------------------------------------------------------%
    \stepcounter{question}
    \begin{question}{\thequestion}{4}

        For the given tree:
        
        \begin{center}
            \begin{tikzpicture}draw (0,5) circle (1pt) node[above] {Proto-indo-european};                
            \end{tikzpicture}
        \end{center}
        
    \end{question}





\end{document}










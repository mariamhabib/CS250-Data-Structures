\documentclass[a4paper,12pt]{book}
\usepackage[utf8]{inputenc}
\title{}
\author{Rachel Morris}
\date{\today}

\usepackage{rachwidgets}
\usepackage{fancyhdr}
\usepackage{lastpage}
\usepackage{boxedminipage}

\newcommand{\laClass}{CS 250\ }
\newcommand{\laSemester}{Fall 2017\ }
\newcommand{\laLab}{Lab 6: Searching and Efficiency\ }

\pagestyle{fancy}
\fancyhf{}
\lhead{\laClass}
\chead{\laSemester}
\rhead{\laLab}
\rfoot{\thepage\ of \pageref{LastPage}}
\lfoot{By Rachel Morris, \footnotesize last updated \today}

\renewcommand{\headrulewidth}{2pt}
\renewcommand{\footrulewidth}{1pt}

\begin{document}

    \chapter*{\laLab} \stepcounter{chapter}

        \section*{Information}
            \paragraph{ Topics: } Searching, intro to efficiency concepts
            \paragraph{ Turn in: } Print out the assignment and work it out on paper.
                Either scan or photograph the assignment once you're done
                and upload it to the Dropbox.

% ----------------------------------------------------------------------
% ----------------------------------------------------------------------
% ----------------------------------------------------------------------
    
    \section{Stepping through code}
    
    \begin{intro}{\ }
        For the following questions, a search algorithm will be given, 
        as well as the inputs. You will need to act as the human computer 
        and step through the algorithm, one command at a time, recording 
        the changes to the variables and stepping through the flow of the function.
        
        For example, if we have the function:
\begin{verbatim}
int FindItem( int arr[], int arraySize, int searchItem )
{	
	for ( int i = 0; i < arraySize; i++ )
	{
		if ( arr[i] == searchItem )
		{
			return i;
		}
	}
	
	return -1;
}
\end{verbatim}

        And using the inputs:
\begin{verbatim}
int pos = FindItem( { 1, 3, 5, 7 }, 4, 5 );
\end{verbatim}

    \end{intro}


\end{document}










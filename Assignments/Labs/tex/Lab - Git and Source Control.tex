\documentclass[a4paper,12pt,oneside]{book}
\usepackage[utf8]{inputenc}
\renewcommand{\chaptername}{Part}

\usepackage{rachwidgets}

\titleformat{\chapter}[display]
  {\bfseries\Large}
  {\filright\MakeUppercase{\chaptertitlename} \Huge\thechapter}
  {1ex}
  {\titlerule\vspace{1ex}\filleft}
  [\vspace{1ex}\titlerule]

\newcommand{\laClass}       {CS 250}
\newcommand{\laSemester}    {Spring 2018}
\newcommand{\laChapter}     {}
\newcommand{\laType}        {Lab}
\newcommand{\laAssignment}  {}
\newcommand{\laPoints}      {5}
\newcommand{\laTitle}       {Git and Source Control}
\newcommand{\laStarterFiles}{Download from GitHub.}
\newcommand{\laTopics}      {Source Control and Git basics}
\setcounter{chapter}{1}
\setcounter{section}{1}
\addtocounter{section}{-1}
\newcounter{question}
\toggletrue{answerkey}
\togglefalse{answerkey}
\input{BASE-HEADER}


    \section{About}

    Source Control is a necessary tool for developers and all programmers should have
    some level of fluency with it. During this semester, you will be storing your projects
    in a repository.

    \section{Requirements}

    Use the documentation provided (class lecture, external reading, asking questions)
    to complete the following:

    \begin{enumerate}
        \item   Create a BitBucket account.
        \item   Create a private repository named ``CS 250''.
        \item   Pull down your CS 250 repository to your local machine.
        \item   Create a ``Labs'' folder.
        \item   Within the Labs folder, create a ``STL Lab'' folder.
        \item   Within the STL Lab folder, copy your code from the STL Lab.
        \item   Add, commit, and push your changes.
        \item   Take a screenshot of your repository front page, and the STL files from the web interface.
                Upload these as part of the lab.
    \end{enumerate}

    

\input{BASE-FOOT}

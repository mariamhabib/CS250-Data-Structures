\documentclass[a4paper,12pt,oneside]{book}
\usepackage[utf8]{inputenc}
\renewcommand{\chaptername}{Part}

\usepackage{rachwidgets}

\titleformat{\chapter}[display]
  {\bfseries\Large}
  {\filright\MakeUppercase{\chaptertitlename} \Huge\thechapter}
  {1ex}
  {\titlerule\vspace{1ex}\filleft}
  [\vspace{1ex}\titlerule]

\newcommand{\laClass}       {CS 250}
\newcommand{\laSemester}    {Spring 2018}
\newcommand{\laChapter}     {}
\newcommand{\laType}        {Lab}
\newcommand{\laAssignment}  {}
\newcommand{\laPoints}      {5}
\newcommand{\laTitle}       {Recursion}
\newcommand{\laStarterFiles}{Download from GitHub.}
\newcommand{\laTopics}      {Writing simple recursive functions}
\setcounter{chapter}{1}
\setcounter{section}{1}
\addtocounter{section}{-1}
\newcounter{question}
\toggletrue{answerkey}
\togglefalse{answerkey}

\title{}
\author{Rachel Singh}
\date{\today}

\pagestyle{fancy}
\fancyhf{}

\lhead{\laClass, \laSemester}

\chead{}

\rhead{\laType\ \laAssignment: \laTitle\ \iftoggle{answerkey}{ KEY }{}}

\rfoot{\thepage\ of \pageref{LastPage}}

\lfoot{\scriptsize By Rachel Singh, last updated \today}

\renewcommand{\headrulewidth}{2pt}
\renewcommand{\footrulewidth}{1pt}

\begin{document}




\footnotesize
~\\ 
\textbf{\laType\ \laAssignment: \laTitle \tab } 
In-class labs are meant to introduce you to a new topic and provide
some practice with that new topic. Labs should be worked on by each
individual student, though asking others for help is permitted.
\underline{Do not} copy code from other sources, and do not give your
code to other students.

\hrulefill
\normalsize 


\section{About}

Recursion can be hard to think in terms of, especially early on.
For this lab, we will step through some recursive functions,
and then you will be challenged to try to design solutions for the others.

~\\ Reference the textbook, reading notes, and lectures for help.

\hrulefill

\subsection{Setting up the project}

Download the starter zip file, \texttt{LAB-RECURSION-INTRO.zip}, off GitHub.
This zip contains the following:

\begin{itemize}
    \item   \texttt{function1.hpp}
    \item   \texttt{function2.hpp}
    \item   \texttt{function3.hpp}
    \item   \texttt{function4.hpp}
    \item   \texttt{function5.hpp}
    \item   \texttt{program\_main.cpp}
    \item   \texttt{Makefile}
\end{itemize}

All of these sources belong in one project and are not separate programs.

The main program has a main menu to allow the user to run one function
at a time. Each of the function files contains stubs for iterative
and recursive functions.

\newpage

% -------------------------------------------------------------------------------------------- %
% -------------------------------------------------------------------------------------------- %
% -------------------------------------------------------------------------------------------- %

\section{Lab specifications}

% -------------------------------------------------------------------------------------------- %
\subsection{Function 1: CountUp}

\begin{lstlisting}[style=code]
void CountUp_Iter( int start, int end )
{
}

void CountUp_Rec( int start, int end )
{
}
\end{lstlisting}

~\\

For this first set of functions, you will write each of these so that
they display the numbers between \texttt{start} and \texttt{end} (inclusive),
going up by 1 each time. The output will look like this:

\begin{lstlisting}[style=output]
CountUp, Iterative:
1	2	3	4	5	6	7	8	9	10

CountUp, Recursive:
1	2	3	4	5	6	7	8	9	10 
\end{lstlisting}

\subsubsection{Iterative version}

The iterative version is pretty easy - write a for-loop and display each number.

\subsubsection{Recursive version}

For the recursive version, follow these steps:

\begin{itemize}
    \item   Display \texttt{start}, plus one tab, with a \texttt{cout} statement.
    \item   \textbf{Terminating case:}
                If the value of \texttt{start} and \texttt{end} are the same,
                then \texttt{return;} out of the function.
    \item   \textbf{Recursive case:}
                Call \texttt{CountUp\_Rec}, passing in \texttt{start+1} and \texttt{end}.
\end{itemize}

Once you get it working, move on to the next function.

\newpage

% -------------------------------------------------------------------------------------------- %
\subsection{Function 2: MultiplyUp}

\begin{lstlisting}[style=code]
void MultiplyUp_Iter( int start, int end )
{
}

void MultiplyUp_Rec( int start, int end )
{
}
\end{lstlisting}

This function is similar to CountUp, except it multiplies the value each time.
The output will end up looking like this:

\begin{lstlisting}[style=output]
MultiplyUp, Iterative:
2	4	16	256	

MultiplyUp, Recursive:
2	4	16	256	
\end{lstlisting}

\paragraph{Terminating case:}
    Stop once \texttt{start} is greater than \texttt{end}. You can use
    \texttt{return;} to leave a function, even though its return-type is void.

\paragraph{Recursive case:}
    Call the function again, passing in $start * start$ and $end$.


\newpage

% -------------------------------------------------------------------------------------------- %
\subsection{Function 3: Factorial}

\begin{lstlisting}[style=code]
int Factorial_Iter( int n )
{
    return -1; // placeholder;
}

int Factorial_Rec( int n )
{
    return -1; // placeholder
}
\end{lstlisting}

For these functions, some value $n$ is passed in. Each of these will calculate
$n!$ by computing $n * (n-1) * (n-2) * ... * 2 * 1$.
For the recursive case, we can think of the calculation recursively:

\begin{center}
    $n! = n * (n-1)!$ \\ and \\ $n! = n * (n-1) * (n-2)!$
\end{center}

\paragraph{Terminating case:}
    End once $n = 1$, returning the value of $n$. (At this point,
    we don't need to go past 1 in our calculation.)

\paragraph{Recursive case:}
    Multiply $n$ by $(n-1)!$ \\
    How do you calculate $(n-1)!$ ? You call \texttt{Factorial\_Rec} again,
    but passing in $n-1$.

    
\begin{lstlisting}[style=output]
Factorial, Iterative:

2! = 2		3! = 6		
4! = 24		5! = 120		
6! = 720		7! = 5040		
8! = 40320		9! = 362880		

Factorial, Recursive:

2! = 2		3! = 6		
4! = 24		5! = 120		
6! = 720		7! = 5040		
8! = 40320		9! = 362880
\end{lstlisting}

\newpage

% -------------------------------------------------------------------------------------------- %
\subsection{Function 4: CountConsonants}

\begin{lstlisting}[style=code]
int CountConsonants_Iter( string text )
{
    return -1; // placeholder
}

int CountConsonants_Rec( string text, int pos )
{
    return -1; // placeholder
}
\end{lstlisting}

For CountConsonants, it will iterate through each letter in the
\texttt{text} parameter, incrementing a counter each time it finds a consonant.

You can use the included function, \texttt{bool IsConsonant( char letter )}
to check each letter.

You can also treat a string as an array of chars, so accessing
\texttt{text$[$0$]$}
would give you the first letter. You can also access the string's
size with \texttt{text.size()}. For instance, if you wanted to display each
letter from a string \texttt{text}, you could use:

\begin{verbatim}
for ( int i = 0; i < text.size(); i++ )
    cout << text[i] << endl;
\end{verbatim}

\paragraph{Terminating case:}
    If you're at the end of the string (that is, \texttt{pos = text.size()}),
    then return 0.

\paragraph{Recursive case:}
    You will have 1 if the current letter (\texttt{text$[$pos$]$}) is a consonant,
    and 0 if the current letter is not.
    Add this value to the consonants 
    You will add \texttt{CountConsonants\_Rec}, passing in text and $pos+1$.


\begin{lstlisting}[style=output]
GetConsonants, Iterative:
 * Consonants in aeiou: 0
 * Consonants in kittens: 5
 * Consonants in development: 7


GetConsonants, Recursive:
 * Consonants in aeiou: 0
 * Consonants in kittens: 5
 * Consonants in development: 7
\end{lstlisting}


\newpage

% -------------------------------------------------------------------------------------------- %
\subsection{Function 5: GetFirstUppercase}

\begin{lstlisting}[style=code]
char GetFirstUppercase_Iter( string text )
{
    return ' '; // placeholder
}

char GetFirstUppercase_Rec( string text, int pos )
{
    return ' '; // placeholder
}
\end{lstlisting}

For this function, it will go through the entire text string.
If it encounters an uppercase letter, it will return that letter.
If the entire string is searched and no uppercase letter is found,
it will return a space char: \texttt{' '}.

For the recursive version, there will be two terminating cases:
if you find an uppercase letter (then return it), and if
you get to the end of the string (return \texttt{' '}).

The recursive case should be similar to with the previous function.

\begin{lstlisting}[style=output]
GetFirstUppercase, Iterative:
 * First upper-case in how are YOU?: 'Y'
 * First upper-case in What?: 'W'
 * First upper-case in where am I?: 'I'
 * First upper-case in no caps: ' '


GetConsonants, Recursive:
 * First upper-case in how are YOU?: 'Y'
 * First upper-case in What?: 'W'
 * First upper-case in where am I?: 'I'
 * First upper-case in no caps: ''
\end{lstlisting}











\end{document}


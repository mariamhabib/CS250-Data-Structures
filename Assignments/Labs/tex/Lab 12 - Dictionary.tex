\documentclass[a4paper,12pt,oneside]{book}
\usepackage[utf8]{inputenc}
\title{}
\author{Rachel Morris}
\date{\today}

\usepackage{rachwidgets}
\usepackage{fancyhdr}
\usepackage{lastpage}
\usepackage{boxedminipage}
\usepackage{fancyvrb}

\newcommand{\laClass}{CS 250\ }
\newcommand{\laSemester}{Fall 2017\ }
\newcommand{\laLab}{Lab 12: Dictionary\ }
\newcounter{question}

\pagestyle{fancy}
\fancyhf{}
\lhead{\laClass}
\chead{\laSemester}
\rhead{\laLab}
\rfoot{\thepage\ of \pageref{LastPage}}
\lfoot{By Rachel Morris, \footnotesize last updated \today}

\renewcommand{\headrulewidth}{2pt}
\renewcommand{\footrulewidth}{1pt}

%\toggletrue{answerkey}
\togglefalse{answerkey}

\begin{document}

    \chapter*{\laLab} \stepcounter{chapter}

        \section{Information}
            \paragraph{ Topics: } Dictionary (aka Hash Table / Associative Array)
            \paragraph{ Turn in: } All source files (.cpp and .hpp).
            \paragraph{ Starter files: } Download on GitHub or D2L.
        
            \paragraph{ Penalties: }
                The following items will negatively impact your score.

                \begin{itemize}
                    \item   \textbf{Program doesn't build}
                        \\ Your program should always build. Programs turned in that don't build will automatically receive a grade of 50\%.{}
                            Additionally, I build your code from the command line in Linux; Your code should be portable. Certain features
                            are allowed in Visual Studio or Windows but don't work for all compilers. \\
                            \footnotesize Avoid: \texttt{\#pragma once}, \texttt{system("pause");}, ignoring filename cases
                            \normalsize 
                    \item   \textbf{Missing source files}
                        \\ If your .hpp or .cpp files are missing, they cannot be graded and will result in a 0\%. Always double-check to make sure you're submitting all your files.
                    \item   \textbf{Visual Studio files}
                        \\ I don't want these. I ONLY want your .hpp and .cpp files. I won't count off if you turn it in, but do me a favor (and help me grade quickly) by not turning in junk files.
                    \item   \textbf{Zipped files}
                        \\ I don't want this. Just submit your source files. I won't count off if you turn in a zip, but when I download assignments they're already zipped so it just makes more work for me.
                \end{itemize}
% ----------------------------------------------------------------------
% ----------------------------------------------------------------------
% ----------------------------------------------------------------------

\renewcommand*\DTstylecomment{\rmfamily\color{green}\textsc}

\begin{framed}
\dirtree{%
.1 Lab 12 - Dictionary/.
.2 lab12\_main{.}cpp.
.2 lab12\_StudentTable{.}hpp.
.2 lab12\_StudentTable{.}cpp.
.2 students{.}csv
    \dots{} \begin{minipage}[t]{6cm} Data input file \end{minipage}.
}
\end{framed}
   
% ----------------------------------------------------------------------
% ----------------------------------------------------------------------
% ----------------------------------------------------------------------

\tableofcontents

\section{About: Dictionaries}

\subsection{Key-value pairs}

If you're familiar with other programming lanugages like Python,
Lua, PHP, or JavaScript, or have used the map STL structure in C++,
you might be familiar with the idea of storing data in an array,
but having a \textbf{key} map to some value, rather than just
an \textbf{index}.

Because of the way a key is associated with a value in the Dictionary,
these structures are often called \textbf{associative arrays}. They
are sometimes also called \textbf{hash tables}, because they use a
hashing function to achieve the key-value pair functionality,
or even a \textbf{map}, because they map a key to a value.

\begin{hint}{Key vs. Index?}
    The \textbf{index} of an element in an array refers to its position
    in the array. If the array is of length $n$, then valid indices
    are $0$ through $n-1$.

    A \textbf{key} is similar to an index, in that you use the key to
    find an element of the array, but the key can be \textit{any data-type}.
    For example, if you had an alphanumeric Employee ID, you could use
    that as a key to point to some Employee element in an array.
\end{hint}

Since the key of a dictionary can be of any data-type, and the value
of a dictionary can also be of any data-type, it can be a useful
structure to give some more meaningful data (than just \textit{index/position})
as a way to map some value.

\subsection{Dictionary basics}

A Dictionary uses an array as its underlying structure. This allows us
to access elements of the array with a $O(1)$ access time.

~\\
\textbf{But if we have a key that points to a value (not an index), how do we still
achieve the $O(1)$ access time?}

We actually achieve this by creating a \textbf{Hash function}, which
converts the key into an index via some sort of mathematical function.

As a simple example, let's say we have the following array:

\begin{center}
    \begin{tabular}{| c | c | c | c | c | c |}
        \hline
        0 & 1 & 2 & 3 & ... & 26
        \\ \hline
        & & & & &
        \\ \hline
    \end{tabular}
\end{center}

And let's say the keys we're using are just letters of the alphabet.
We could easily map these values to each number. The letter 'A' has an ASCII code of
65, 'B' is 66, and so on, so we could essentially create a Hash
function like this:

\begin{verbatim}
int Hash( char key )
{
    return int( key ) - 65;
}
\end{verbatim}

Then if we called \texttt{Hash('A')}, we would get a value of 0.
For \texttt{Hash('D')}, we would get a value of 3.

\subsection{Hash functions}

Usually, we will want something a bit more sophisticated than just
single letters as the key for our Dictionaries. With a hash
function, there is some probability that two or more different keys
will generate the same index. This is called a \textbf{collision}.
There are several ways we can deal with collisions in a Dictionary.

\subsubsection{Simple Hash function}

For now we will think about integer keys, to make it simpler to understand
the basic hashing concepts. Let's say a key can be any integer value
of any length, and we have an array to fill up - let's say it's only
length 10 for now.

\begin{center}
    \begin{tabular}{| c | c | c | c | c | c | c | c | c | c |}
        \hline
        0 & 1 & 2 & 3 & 4 & 5 & 6 & 7 & 8 & 9
        \\ \hline
        & & & & & & & & &
        \\ \hline
    \end{tabular}
\end{center}

And for our hash function, we start with a simple one where we make the
key the index.

\begin{verbatim}
int HashA( int key )
{
    return key;
}
\end{verbatim}

This would be OK as long as the keys were 0 through 9, but if the
key is 10 or more, we will run into a problem. OK - we can handle this,
let's just wrap around if we go past the bounds of the array. For example,
if we have 10, we wrap around to 0. If we have 11, we wrap around to 1,
and so on.

\begin{verbatim}
int HashB( int key )
{
    // SIZE = 10
    return key % SIZE; 
}
\end{verbatim}

This will work alright... now our keys can be any length, because
the modulus operation will restrict the index to being between 0 and $SIZE$.

\newpage
However, let's say all our keys end up being multiples of 5...

\begin{itemize}
    \item \texttt{Table size: 10. Hash(key) \{ return key \% 10;\}}
    \item \texttt{Insert item at key 5: Index is 5}
    \item \texttt{Insert item at key 10: Index is 0}
    \item \texttt{Insert item at key 15: Index is 5 \tab COLLISION!}
    \item \texttt{Insert item at key 20: Index is 0 \tab COLLISION!}
    \item \texttt{Insert item at key 25: Index is 5 \tab COLLISION!}
    \item \texttt{Insert item at key 30: Index is 0 \tab COLLISION!}
\end{itemize}

We will talk about how to handle collisions in a little bit, but for now
it is good to note that a general good design rule for Dictionaries
is to make sure the size of your array is a \textbf{prime number}.

\begin{itemize}
    \item \texttt{Table size: 13. Hash(key) \{ return key \% 13;\}}
    \item \texttt{Insert item at key 5: Index is 5}
    \item \texttt{Insert item at key 10: Index is 10}
    \item \texttt{Insert item at key 15: 15 \% 13 = 2; \tab Index is 2}
    \item \texttt{Insert item at key 20: 20 \% 13 = 7; \tab Index is 7}
    \item \texttt{Insert item at key 25: 25 \% 13 = 12; \tab Index is 12 }
    \item \texttt{Insert item at key 30: 30 \% 13 = 4; \tab Index is 4 }
\end{itemize}

See? Without doing any extra work, we can avoid collisions much easier
by simply making sure our table (array) size is a prime number.


\subsection{Collisions}

With enough data, collisions are bound to happen at some point. A collision
is when the Hash function generates an index for a key, but that index
is already taken by some other data.

\subsubsection{Linear probing}

With linear probing, the solution to a collision is simply to go forward
by one index and see if \textit{that} position is free. If not,
continue stepping forward by 1 until an available space is found.

\subsubsection{Quadratic probing}

With quadratic probing, we will keep steping forward until we find
an available spot, just like with linear probing. However, the amount
we step forward \textit{by} changes each time we hit a collision on an insert.

\begin{itemize}
    \item For the first collision, the index $1^{2}$ position over is tested. If
    that causes a collision, then we try a second time.

    \item For the second collision, we try $2^{2}$ positions over (from the original
    index we generated). If this causes a collision, then we try a third time.

    \item For the third collision, we try $3^{2}$ positions over (from the original
    index we generated). And so on...
\end{itemize}

So, we square the amount of times we've hit a collision, and use that
to offset the original index.

\subsubsection{Double hashing}

With double hashing, we have \textit{two} hash functions. If the first
Hash function returns an index that is already taken, then we
use the \textbf{second Hash function} on the key to generate an offset.
Then, our new index will be $Hash(key) + Hash2(key) \% $ TABLE\_SIZE.

\subsubsection{Clustering}

Clustering is a problem that arises, particularly when using linear probing.
If we're only stepping forward by 1 index every time there's a collision,
then we tend to get a Dictionary table where all the elements are clustered
together in groups...

\begin{center}
    \begin{tabular}{| c | c | c | c | c | c | c | c | c | c |}
        \hline
        0 & 1 & 2 & 3 & 4 & 5 & 6 & 7 & 8 & 9
        \\ \hline
        & A & B & C & D & & & X & Y & Z
        \\ \hline
    \end{tabular}
\end{center}

Because of this, the more full the Dictionary gets, the less efficient
it becomes when using linear probing. Quadratic probing helps mitigate this,
because it jumps forward by a quadratic amount when searching for an
available index.

\section{Lab specifications}

\subsection{Your program should always build}

\begin{error}{\ }
Before you write any code... build the program.

~\\
After implementing one function... build the program.

~\\
It is much easier to debug your syntax and logic errors if the scope
of what has changed is very small.
\end{error}

\subsection{About the program}

For this program, a list of students and their student IDs are
loaded in.
They will be stored in a Dictionary/Hash Table, and you will need to
implement the Insert, Hash, and collision functions.

When you run the program, you will have the option of loading in
a text file of users or generating them. You can use either method.

\begin{lstlisting}[style=output]
Choose how to create students.

1. Load students from file
2. Generate students
>> 2
100 students generated

Choose the method of collision handling for the table.

1. Store with Linear Probing
2. Store with Quadratic Probing
3. Store with Double Hashing
4. Exit
>> 1
Collision method: 0

Output table to: out_linear.txt

0.   Student ID: 786 	 Student Name: izu
1.   Student ID: 655 	 Student Name: oniwa
(ETC.)
\end{lstlisting}

After you choose the collision method, it will display the results
to the screen as well as to an output text file.

\begin{center}
    \begin{tabular}{l l l}
        Linear probing & out\_linear.txt \\
        Quadratic probing & out\_quadratic.txt \\
        Double hashing & out\_double.txt    
    \end{tabular}
\end{center}

%--------------------------------------------%

\subsection{The StudentTable class}

The class declaration for StudentTable is as follows:

\begin{lstlisting}[style=code]
class StudentTable
{
    public:
    StudentTable();

    void SetCollisionMethod( CollisionMethod cm );
    void Insert( int studentId, const string& name );
    void Output( const string& filename );

    private:
    int HashFunction( int key );

    /* Linear probing */
    // You can use my function prototype or write your own.
    int LinearProbe( int key );

    /* Quadratic probing */
    // You can use my function prototype or write your own.
    int QuadraticProbe( int key, int& addValue );

    /* Double hashing */
    // You can use my function prototype or write your own.
    int HashFunction2( int key );

    Student m_students[TABLE_SIZE];

    CollisionMethod m_collisionMethod;
};
\end{lstlisting}

Of particular note, there is

\begin{center}
    \texttt{
        Student m\_students[TABLE\_SIZE];
    }
\end{center}

which is where the data will be stored. This is the table. We will
take some input key - the student ID - and convert it into a valid index.
If the index generated is already taken by some other data, then we
use our collision method to generate a new index until we find a free position.

There is also

\begin{center}
    \texttt{
        CollisionMethod m\_collisionMethod;
    }
\end{center}

With the \texttt{CollisionMethod} type being declared in the StudentTable header file...

\begin{center}
    \texttt{
        enum CollisionMethod \{ LINEAR, QUADRATIC, DOUBLE \};
    }
\end{center}

In order to check which collision method to call within \texttt{StudentTable},
you will check what the value of the \texttt{m\_collisionMethod} variable is
set to like this...

\begin{verbatim}
    if ( m_collisionMethod == LINEAR )
    {
        // ... do stuff ...
    }
\end{verbatim}


%--------------------------------------------%

\subsection{StudentTable::Insert}

\paragraph{Parameters:}

\begin{itemize}
    \item   studentId, an integer
    \item   name, a const string reference
\end{itemize}

\paragraph{Return type:} void

\paragraph{Functionality:}

\begin{enumerate}
    \item   Create a new local integer variable named \texttt{index}.
    \item   Call the \texttt{HashFunction} function, passing in the
        \texttt{studentId} as the key. Store the result of \texttt{HashFunction}
        in the \texttt{index} variable.
    \item   Access the element of \texttt{m\_students} at this index. Set...
    \begin{enumerate}
        \item   its \texttt{key} to the \texttt{studentId} parameter.
        \item   its \texttt{value} to the \texttt{name} parameter.
    \end{enumerate}
\end{enumerate}

%--------------------------------------------%
\hrulefill

\subsection{StudentTable::LinearProbe}

\paragraph{Parameters:}

\begin{itemize}
    \item   key, an integer
\end{itemize}

\paragraph{Return type:} int, the index after a linear probe

\paragraph{Functionality:}
Return the value of the \texttt{key}, plus 1.

%--------------------------------------------%
\hrulefill

\subsection{StudentTable::QuadraticProbe}

\paragraph{Parameters:}

\begin{itemize}
    \item   key, an integer
    \item   addValue, an integer reference
\end{itemize}

\paragraph{Return type:} int, the index after a quadratic probe

\paragraph{Functionality:}
Create a local integer variable named \texttt{quadratic}. Assign it
the value of \texttt{addValue * addValue} (the addValue squared).

Then, increment \texttt{addValue} by 1.

Finally, return \texttt{key + quadratic}.

%--------------------------------------------%
\hrulefill

\subsection{StudentTable::HashFunction2}

\paragraph{Parameters:}

\begin{itemize}
    \item   key, an integer
\end{itemize}

\paragraph{Return type:} int, the index after this hash   

\paragraph{Functionality:}

Return the equation: \texttt{7 - ( key \% 7 )}




%--------------------------------------------%
\hrulefill

\subsection{StudentTable::HashFunction}

\paragraph{Parameters:}

\begin{itemize}
    \item   key, an integer
\end{itemize}

\paragraph{Return type:} int, the resulting index

\paragraph{Functionality:}

The HashFunction is the ``home base", that will make sure we have a valid
index, and call the appropriate collision method until we have one.
It will also keep track of the variables used for the Quadratic and Double Hash methods,
if needed.

For the \texttt{m\_students} array, everything has been initialized so that
if it is not currently in use, that element's key is -1.

\begin{enumerate}
    \item   Create a new local integer variable named \texttt{index}.
        Assign it the value of \texttt{key \% TABLE\_SIZE}.
    \item   Create a new local integer variable named \texttt{originalIndex}.
        Assign it the value of \texttt{index}.

    \item   Create a new local integer variable named \texttt{quadraticAdd},
        and initialize it to 1.
    \item   Create a new local integer variable named \texttt{doubleHashAdd},
        and initialize it to 0.

    \item   Create a while loop. It will continue looping while the
        \texttt{m\_students} element at position \texttt{index} has a
        \texttt{key} that isn't -1. Within the loop...
    \begin{enumerate}
        \item   If the collision method is LINEAR...
            \begin{enumerate}
                \item   Call the \texttt{LinearProbe} function,
                    passing in the current \texttt{index}. Store the
                    result in the \texttt{index} variable.
                \item   We need to make sure the index is within the array
                    bounds, so you will also take the \texttt{index \% TABLE\_SIZE},
                    and assign the result back into \texttt{index}.
            \end{enumerate}
            
        \item   If the collision method is QUADRATIC...
            \begin{enumerate}
                \item   Call the \texttt{QuadraticProbe} function,
                    passing in the \texttt{originalIndex} and the
                    \texttt{quadraticAdd} variables. Store the result in the
                    \texttt{index} variable.
                \item   We need to make sure the index is within the array
                    bounds, so you will also take the \texttt{index \% TABLE\_SIZE},
                    and assign the result back into \texttt{index}.
            \end{enumerate}
            
        
        \item   If the collision method is DOUBLE...
            \begin{enumerate}
                \item   Call the \texttt{HashFunction2} function, passing in
                    the \texttt{originalIndex}. \textbf{Add} the result
                    onto the \texttt{doubleHashAdd} variable.
                \item   Set the \texttt{index} to
                    \texttt{( originalIndex + doubleHashAdd ) \% TABLE\_SIZE}
            \end{enumerate}
            
    \end{enumerate}

    \item   After the while loop has finished, we have a valid index. Return the \texttt{index}.
    
\end{enumerate}

\hrulefill

\subsection{Example output}

Once the program has been run, your output text files will look like this:

\begin{lstlisting}[style=textfile]
0.   Student ID: 786 	 Student Name: izu
1.   Student ID: 655 	 Student Name: oniwa
2.   Student ID: 391 	 Student Name: ecew
3.   Student ID: 653 	 Student Name: otaf
4.   Student ID: 1052 	 Student Name: ayekut
5.   Student ID: 529 	 Student Name: iseno
6.   Student ID: 397 	 Student Name: erum
7.   
8.   
9.   Student ID: 795 	 Student Name: iguz
10.  Student ID: 272 	 Student Name: arizabo
11.  
12.  
13.  
14.  Student ID: 145 	 Student Name: etavu
15.  Student ID: 1063 	 Student Name: ezekali
16.  Student ID: 540 	 Student Name: oqovi
17.  Student ID: 669 	 Student Name: ejumu
18.  Student ID: 669 	 Student Name: oyu
19.  Student ID: 150 	 Student Name: arukaki
20.  
21.  Student ID: 414 	 Student Name: ebox
(AND SO ON...)
\end{lstlisting}

Make sure to investigate all three output files, out\_linear.txt,
out\_quadratic.txt, and out\_double.txt, and look at how much
clustering occurs, or otherwise how spaced out the elements are.





\end{document}










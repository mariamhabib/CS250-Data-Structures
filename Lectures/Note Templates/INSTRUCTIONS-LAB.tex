\footnotesize
~\\ 
\textbf{\laType\ \laAssignment: \laTitle \tab } 
In-class labs are meant to introduce you to a new topic and provide
some practice with that new topic. Labs should be worked on by each
individual student, though asking others for help is permitted.
\underline{Do not} copy code from other sources, and do not give your
code to other students. Students who commit or aid in plagiarism will
receive a 0\% on the assignment and be reported.

\hrulefill
\normalsize 

    \subsection*{Information}
    
        \paragraph{Topics:} \laTopics

        \paragraph{Turn in:}
            Turn in all source files - .cpp, .hpp, and/or .h files.
            \textbf{Do not turn in Visual Studio files}.

        \paragraph{Starter files:} \laStarterFiles

        \paragraph{Grading:}
            Grading is based on completion, if the program functions as intended,
            and absense of errors. Programs that don't build will receive a 50\%.{}
            Build errors, runtime errors, logic errors, memory leaks, and ugly code will reduce your score.

        \paragraph{Building and running:} If you are using Visual Studio,
            make sure to run \underline{with} debugging. (Don't run without debugging!)
            Using the debugger will help you find errors. \\
            To prevent a program exit, use this before \texttt{return 0;}
\begin{verbatim}
cin.ignore();
cin.get();
\end{verbatim}

        \tableofcontents

    \newpage
